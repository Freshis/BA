\documentclass[12pt,titlepage]{article}

% proper umlauts
\usepackage[utf8]{inputenc}
\usepackage[T1]{fontenc}

% german
\usepackage[ngerman]{babel}

% AMS math for formulae
\usepackage{amsmath}
\usepackage{amssymb}
\usepackage{amsfonts}

% graphics
\usepackage{graphicx}
\usepackage{color}
\usepackage{float} 
\usepackage{lmodern}
\usepackage{mathrsfs}
\usepackage{caption}
\usepackage{tikz}
\usetikzlibrary{positioning}

% referieren
\usepackage{hyperref}

% newcommands
\newcommand{\col}[2][red]{\textcolor{#1}{#2}}
\newtheorem{defi}{Definition}[section]

\begin{document}
\begin{titlepage}
\pagestyle{empty}
 \centering
 {\LARGE Technische Universität Dortmund \par}
 \vspace{0.5cm}
 {\LARGE Fakultät Mathematik \par}
 \vspace{1cm}
 {\Large Bachelorarbeit Technomathematik: Numerik \par}
 \vspace{2cm}
 {\huge\bfseries Die Radialen-Basisfunktion-Finite-Differenzen-Methode und die Stabilisierung mittels Flux-Correction-Tools angewendet auf die Transportgleichung \par}
 \vspace{2cm}
 {\Large Wintersemester 2017/2018 \par}
 \vspace{0.5cm}
 {\large \today\par}
 \vspace{1cm}
 {\Large Professor: Prof. Dr. Stefan Turek \par}
 \vspace{0.5cm}
 {\Large Betreuer: Dr. Andriy Sokolov \par}
 \vfill
 \begin{flushright}
  {Alexander Westermann \par}
  {Matrikel-Nummer: 175180 \par}
  \vspace{0.5cm}
  {alexander.westermann@tu-dortmund.de \par}
  {7.Semester \par}
  {B. Sc. Mathematik \par}
 \end{flushright}
\end{titlepage}
\pagebreak
\pagenumbering{Roman}
 \tableofcontents
\pagebreak
\listoffigures
\pagebreak
\pagenumbering{arabic}
\setcounter{page}{1}
\section{Einleitung}
In der heutigen Wissenschaft werden komplexe Algorithmen mit unterschiedlichen Verfahren verwendet. Durch die Komplexität der Verfahren und der Digitalisierung werden zumeist die Lösungen der Methoden numerisch am Computer approximiert. Einige Beispiele dafür sind die numerische Approximation von Integralen durch \textit{Quadraturformeln}, das Annähern der Lösungen von Fixpunktgleichungen durch den \textit{Newton-Algorithmus} oder das iterative Lösen eines Gleichungssystem mittels dem \textit{CG-Verfahren}. Mit solchen grundlegenden Approximationsmethoden lassen sich komplexere Probleme durch die Numerik realisieren. Beispiele für solche Probleme stellen die Approximationsaufgaben von gewöhnlichen und partiellen Differentialgleichungen dar. Einige naturwissenschaftliche Anwendungsbeispiele, welche durch Differentialgleichungen beschrieben werden, sind die \textit{Newtonsche Himmelsmechanik}~\cite{deuflhard2013gewohnliche}, welche zur Bahnberechnung von Satelliten oder Planetoiden verwendet wird, die \textit{chemische Reaktionskinetik}~\cite{deuflhard2013gewohnliche}, welche in der heutigen industriellen Verfahrenstechnik eine wichtige Rolle spielt, oder allgemeine Simulationsvorgänge. Beispiele, in denen eine Simulation berechnet wird, sind beispielsweise die Simulation der \textit{Flachwassergleichung} oder der \textit{Transportgleichung}, welche in dieser Arbeit in den Fokus gesetzt und simuliert wird. Hierzu wird die \textit{Transportgleichung} im Kapitel \ref{sec:transglei} vorgestellt.\\
Zur Visualisierung der Lösung der \textit{Transportgleichung} muss diese durch numerische Verfahren approximiert werden. Die klassische \textit{Finite-Differenzen-Methode}(FDM) oder die \textit{Finite-Elemente-Methode}(FEM) können diesbezüglich verwendet werden. Diese Methoden sind in der Industrie als Lösungesverfahren für Differentialgleichungen weit verbreitet, wobei beide Verfahren ihre Vor-/ und Nachteile haben. Die Vorteile der FDM liegt in der Einfachheit der Implementierung, wohingegen der Vorteil der FEM in der Flexibilität der Gittergenerierung des zugrunde liegenden Gebiets liegt. Dennoch sind beide Verfahren nach einer festen Wahl des Gitters stark abhängig von diesem Gitter, welches zu einer sehr langsamen Konvergenzgeschwindigkeit der Verfahren in numerisch instabilen Gebieten führen könnte. Um dieses Problem zu umgehen wird in Kapitel \ref{sec:RBF-FD} eine Methode vorgestellt, welche ohne Gittergenerierung ein Differentialgleichungsproblem auf einem beliebigen Gebiet approximieren kann. Diesbezüglich wird durch Kombination der FDM und einer speziell gewählten Wahl von Basisfunktionen eine neue gitterfreie Lösungsmethode hergeleitet.\\
Die Approximation von Differentialgleichungen kann für unterschiedliche Probleme unterschiedliche Eigenschaften aufweisen. Manche Annäherung sind im Gegensatz zu der analytischen Lösung stark diffusiv, andere stark oszillierend. Um Letzteres entgegenzuwirken wird in Kapitel \ref{sec:FCT} ein Stabilisierungsschema vorgestellt, welches durch eine Addition einer künstlichen Diffusion eine nicht-oszillierende Lösung der Differentialgleichung liefert. Hierzu wird dieses Schemata anhand der in Kapitel \ref{sec:RBF-FD} gelösten \textit{Transportgleichung} vorgestellt, da die Lösung dieses Differentialgleichungsproblem oszillierende Merkmale aufweist.\\
Abschließend wird in Kapitel \ref{sec:NumVers} die gitterfreie Approximationsmethode und das Stabiliserungsschema durch numerische Versuche analysiert. In den Versuchen wird speziell auf die \textit{Transportgleichung} eingegangen und das Verhalten der Lösung mit und ohne Stabilisierung gegenübergestellt.
\pagebreak
\section{Transport-Gleichung}\label{sec:transglei}
Die Aussagen dieses Kapitels gehen auf die Ausarbeitung von van Kan und Segal \cite{segal2013numerik} zurück.\\
In der Mathematik gibt es viele unterschiedliche partielle Differentialgleichung, wobei der Großteil der partiellen Differentialgleichungen zweiter Ordnung sind. Diese sind hauptsächlich in elliptische, parabolische und hyperbolische Gleichungen klassifiziert. Einfache Beispiele wären hierfür die elliptische Laplace-Gleichung, parabolische Wärmegleichung und die hyperbolische Wellengleichung. Zu den Letzteren wird auch die sogenannte Transportgleichung gezählt, obwohl diese nicht exakt die Bedingungen für die hyperbolische Klassifizierung erfüllt. Dieses zeitabhängige Problem beschreibt den \grqq Transport\grqq~einer Funktion in einem bestimmten $n$-dimensionalen Gebiet $\Omega$.\\
Die allgemeinste Form der Transportgleichung wird \textit{Erhaltungsform} genannt und lautet in einer Dimension
\begin{align}
 \frac{\partial u}{\partial t}+\frac{\partial f(u)}{\partial x}=g(u,x,t)\label{1d}.
\end{align}
Hierbei stellen $u:\Omega\subset\mathbb{R}^d\to\mathbb{R}$ die zu transportierende Funktion und $f$ den sogenannten $Flussvektor$ dar. Die Funktion $g$ ist abhängig von der Funktion $u$, dem Standort $x$, und der Zeit $t$. Wenn die Abhängigkeit zum Standort und zur Zeit nicht vorhanden ist, nennt mann dieses Problem \textit{autonom}. Desweiteren bezeichnet man das Problem als \textit{homogen}, wenn $g(u,x,t)=0$ ist.\\
Betrachtet man nun die mehrdimensionale, homogene $Erhaltungsgleichung$ der Art
\begin{align}
 \frac{\partial u}{\partial t}+\nabla\cdot\textbf{f}(u)=0,
\end{align}
so kann man nun für den $Flussvektor$ $\textbf{f}(u)=\textbf{v} u$ einsetzen und somit folgt die Gleichung
\begin{align}
 \frac{\partial u}{\partial t}+\nabla \cdot (\textbf{v}u)=0.\label{2d}
\end{align}
Der Vektor $\textbf{v}$ repräsentiert hier nun den, nicht notwendig kontsanten, Geschwindigkeitsvektor und u die zu transportierende, skalare Lösung. Nun kann \eqref{2d} durch die Produktregel, bezüglich der Divergenz, geschrieben werden als
\begin{align}
  \frac{\partial u}{\partial t}+\textbf{v}\cdot\nabla u + u \cdot \nabla \cdot\textbf{v}=0.
\end{align}
Wenn nun noch das Vektorfeld inkrompressibel ist (also $\nabla \cdot \textbf{v}=0$), folgt die Gleichung
\begin{align}
 \frac{\partial u}{\partial t}+\textbf{v}\cdot\nabla u=0,\label{eq:transport3}
\end{align}
welche auch \textit{Advektions- oder Konvektions-Gleichung} genannt wird.\\
Eine weitere nennenswerte Form, welche sich aus der \textit{Erhaltungsgleichung} ableiten lässt, ist die \textit{Kovenktions-Diffusions-Gleichung}:
\begin{align}
 \frac{\partial u}{\partial t}+\nabla \cdot \textbf{v}u = \epsilon\Delta u\label{diffusionsgleichung}
\end{align}
Im Gegensatz zu der \textit{Konvektions-Gleichung} tritt in \eqref{diffusionsgleichung} auch Diffusion mit einem Diffusionskoeffizienten $\epsilon$ auf. Wenn nun $\epsilon$ sehr klein wird, dominiert das erste-Ordnung Glied die Gleichung und wird somit schwierig numerisch zu lösen.
\pagebreak
\section{RBF-FD}\label{sec:RBF-FD}
Die Radiale-Basisfunktion-Finite-Differenzen-Methode (RBF-FDM) behandelt die Approximation gewöhnlicher und partieller Differentialgleichungen. Im Gegensatz zu den klasischen Verfahren, wie zum Beispiel der allgemeinen Finite-Differenzen-Methode oder der Finiten-Elemente-Methode, wird bei dieser Methodik kein vorgeschriebenes Gitter benötigt und ist somit unabhängig von dem gegebenen Raum. Allgemein werden in diesem Kapitel die radialen Basisfunktionen vorgestellt und anhand der Interpolationsaufgabe mittels radialen Basisfunktionen näher erläutert. Dann wird das Verfahren erläutert, indem die Basisfunktionen im Zusammenhang mit dem Finiten-Differenzenverfahren gebracht werden. Abschließend wird die RBF-FD-Methode an das zugrunde liegende Problem der Transportgleichung angewendet und Vorbereitungen für die numerische Implementierung gegebene. Als erstes wird nun die Finite-Differenzen-Methode vorgestellt.
\subsection{Einführung in die Finite-Differenzen-Methode}\label{sec:FD}
Dieses Kapitel liegt den Arbeiten von Martin Burger \cite{burger2006numerik} und Dietrich Braess \cite{braess2013finite} zu Grunde.\\
In der numerischen Mathematik ist das Approximieren gewöhnlicher und partieller Differentialgleichung ein Hauptforschungsgebiet. Hierzu sind im Laufe der Zeit viele verschiedene Verfahren entwickelt und weiterentwickelt worden. Einer der ältesten  und einfachsten Verfahren ist die sogenannte \textit{Finite-Differenzen-Methode} (FDM), welche zu der Klasse der Differenzenverfahren gehört. Bei diesem Verfahren nähert man die Ableitung einer Funktion $u(x)$ mittels Differenzenquotienten an:
\begin{align}
 \frac{\partial u(x)}{x}=\lim\limits_{h \rightarrow 0} \frac{u(x+h)-u(x)}{h}
\end{align}
Wenn man nun $h$ klein genug wählt($h\ll1$), ist dies eine gute Approximation der Ableitung:
\begin{align}
 \frac{\partial u(x)}{x}\approx \frac{u(x+h)-u(x)}{h}\label{eq:vordiff}
\end{align}
Die Gleichung \eqref{eq:vordiff} wird in der Literatur als \textit{Vorwärtsdifferenzenquotient} bezeichnet und wird durch das Zeichen $D^+u$ abgekürzt. Bei diesem Operator spricht man von einem Differnzenquotient 1. Ordnung, da dieser die erste Ableitung approximiert. Weitere geläufige Arten 1. Ordnung sind der \textit{Rückwärtsdifferenzenquotient}
\begin{align}
 D^-u= \frac{u(x)-u(x-h)}{h}
\end{align}
und der \textit{Zentrale Differenzenquotient}, welcher eine Linearkombination des Vorwärts- und Rückwärtsdifferenzenquotienten darstellt:
\begin{align}
 D^cu=\frac{1}{2}(D^+u+D^-u)= \frac{u(x+h)-u(x-h)}{2h}
\end{align}
Zur Berechnung der Fehlerordnung verwendet man die \textit{Taylor-Entwicklung} (TE), welche allgemein
\begin{align}
 u(x\pm h)=u(x)\pm hu^{(1)}(x)+\cdots+\frac{(\pm h)^n}{n!}u^{(n)}(x)+u^{(n+1)}(x+\theta h)
\end{align}
mit $\theta\in(0,1)$ lautet. Wendet man nun die Taylorentwicklung zur Bestimmung der Konsistenzordnung auf den Vorwärtsdifferenzenquotient an, so berechnet man:
\begin{align}
 |u^{'}(x)-D^+u(x)|&=|u^{(1)}-\frac{u(x+h)-u(x)}{h}|\nonumber\\&\overset{TE}{=}|-\frac{h}{2}u^{(2)}(x+\theta h)|\le\frac{1}{2} h\|u^{(2)}\|_{\infty}
\end{align}
Hierbei stellt $\|\cdot\|_{\infty}$ die Maximumsnorm einer Funktion dar, welche definiert wird als
\begin{align}
 \|u\|_{\infty}=\underset{x\in\Omega}{max}\{|f(x)|\}.\nonumber
\end{align}
Somit ist der Vorwärtsdifferenzenquotient $\mathcal{O}(h)$ und besitzt eine Fehler von 1. Ordnung. Mit der gleichen Methode berechnet man auch die Fehlerordnungen der anderen beiden Verfahren und bekommt heraus, dass der Fehler des Rückwärtsdifferenzenquotient auch $\mathcal{O}(h)$ und der Fehler vom Zentralen-Differenzenquotienten sogar $\mathcal{O}(h^2)$ und folglich eine Fehlerordnung von 2 besitzt.\\
Im mehrdimensionalen Fall werden somit Ableitungen in jeder Dimension durch solche Differenzenquotienten angenähert. Ein Beispiel im Zweidimensionalen ist die Approximation des zweidimensionalen Laplace-Operators
\begin{align}
 \Delta u(x,y)=\nabla\cdot\nabla u(x,y)=\frac{\partial^2u}{\partial x^2}+\frac{\partial^2u}{\partial y^2},
\end{align}
welcher bei elliptischen Problemen häufig vorkommt. Dieser wird zum Beispiel durch den folgenden Differenzenquotienten mit äquidistanter Gitterweite in beiden Richtungen angenähert:
 \small
\begin{align}
 \Delta u=\frac{u(x+h,y)+u(x-h,y)+u(x,y+h)+u(x,y-h)-4u(x,y)}{h^2}+\mathcal{O}(h^2)\nonumber
\end{align}
\normalsize
Dieser Operator setzt sich zusammen aus den zentralen Differnzenquotient zweiter Ordnung für beide Raumrichtungen und hat bezüglich dem Laplace-Operator eine Fehlerordnung von 2. Bei diesem Operator werden für die Berechnung eines einzelnen Funktionwertes fünf Auswertungen benötigt, sodass man hier auch von einem 5-Punkte-Stern-Operator spricht.\\\\
\textbf{Diskretisierung}\\\\
Für die FDM benötigt man Auswertungsstellen, um, wie oben gesehen, den Differentialoperator approximieren zu können. Hierbei legt man ein Gitter auf ein Gebiet $\Omega$, auf dem man die PDG lösen möchte. Verwendet man ein äquidistantes Gitter ist der Umgang der Differenzenquotienten leicht handhabbar und die Implementierung ist nicht sehr aufwändig. Der Nachteil einer solchen Gitterierung ist die Unflexibilität in Gebieten von $\Omega$, in denen eine genauere Auswertung erforderlich ist, um die dort gegebenen Eigenschaften bestmöglich approximieren zu können.\\
Sei nun einfachheitshalber das Gebiet $\Omega\subset\mathbb{R}^2$. Wählt man eine äquidistante Gittergenerierung mit dem Abstand $h$, so erhält man einen diskretisierten Raum:
\begin{align}
 \Omega_h&=\{(x,y)\in\Omega|x=kh,~y=lh~\text{mit}~k,l\in\mathbb{Z}\}\nonumber\\
 \partial\Omega_h&=\{(x,y)\in\partial\Omega|x=kh~\text{oder}~y=lh~\text{mit}~k,l\in\mathbb{Z}\}\nonumber
\end{align}
Eine Beispieldiskretisierung mit einem äquidistanten Gitter ist in Abbildung \ref{fig:diskGit} dargestellt.
\begin{figure}[ht]
	\centering
	\includegraphics[width=0.4\textwidth]{figures/Diskretisierung.png}
	\caption[Diskretiesierung]{äquidistantes Gitter auf einem Rechteck}
	\label{fig:diskGit}
\end{figure}
Durch die Diskretisierung erhält man nun auch einen diskreten Lösungsvektor $u_h$, welcher in der $i$-ten Zeile die skalare Lösung zum Knoten $i$ besitzt. Dieser ist auf dem diskreten Gebiet $\overline{\Omega}_h=\Omega_h\cup\partial\Omega_h$ definiert. Betrachtet man nun ein System mit Dirichlet-Randbedingungen der Art
\begin{align}
 Lu&=f~~\text{in}~\Omega\nonumber\\
 u&=g~~\text{auf}~\partial\Omega\nonumber,
\end{align}
so gilt nach der Diskretisierung:
\begin{align}
 L_h\{u_h\}&=f_h~~\text{in}~\Omega_h\nonumber\\
 u_h&=g_h~~\text{auf}~\partial\Omega_h\label{eq:allgGleich}
\end{align}
Nun sind die Approximationsmethoden, wie zum Beispiel der oben vorgestellte Vorwärtsdifferenzenquotient, anwendbar und man kann das Gleichungssystem \eqref{eq:allgGleich} schreiben als Linearkombination der Auswertungsstellen der Art:
\begin{align}
 \sum_{i\in I}w_iu_{ij}=f_j~~~~\forall j\in I\label{eq:fd-allge}
\end{align}
In dieser Linearkombination beschreibt $I$ die Indexmenge der Knoten, $u_{ij}$ die Knotenauswertungen und $f_j$ die Auswertung der rechten Seite. Mit $w_i$ werden die jeweiligen Gewichte beschrieben, welche sich aus dem FD-Verfahren bestimmen lassen. Die Gleichung~\eqref{eq:fd-allge} kann in ein lineares Gleichungssystem der Form
\begin{align}
 A_hu_h=f_h
\end{align}
geschrieben werden. Dieses System kann nun durch klassische iterative Verfahren, wie das \textit{Gauß-Seidel-} oder \textit{Jacobi-Verfahren}, oder durch Lösungsalgorithmen, welche speziell auf das Lösen partieller Differentialgleichungen entwickelt wurden, wie das \textit{Mehrgitter-Verfahren}, gelöst werden. Durch eine Interpolation des ermittellten Lösungsvektors kann nun eine Lösungsfunktion angegeben werden, welche somit auch graphisch anzeigbar wäre. Somit wird nun im nächsten Kapitel eine Interpolationsaufgabe vorgestellt.
\subsection{RBF-Interpolation}\label{sec:rbf_inter}
Die Aussagen dieses Kapitels führen auf die Arbeit von Grady B. Wright \cite{wright2003radial} zurück, sodass  hier nur kurz die Thematik der radialen Basisfunktionen angeschnitten wird.\\
In der Numerik ist die Interpolation eine wichtige Aufgabe, um aus einzelnen Funktionswerten die Zielfunktion zu approximieren. Hierbei haben sich schon viele Methoden, wie die \textit{Polynom-} oder \textit{Fourier-Interpolation}, für die eindimensioanle Interpolationsaufgabe in der Wissenschaft etabliert. Alle Verfahren basieren auf die gleiche Grundidee: Zu einer gegebenen Menge an Stützstellen $\{x_i\}_{i=1}^n$ und deren zugehörigen Funktionsauswertungen $\{f_i\}_{i=1}^n$ werden jeweilige Basisfunktionen $\{\phi_i(x)\}_{i=1}^n$ so gewählt, dass die Interpolierende die lineare Form
\begin{align}
 s(x)=\sum_{i=1}^n\lambda_i\phi_i(x)
\end{align}
besitzt. In diesem Zusammenhang stellen die $\lambda_i$ die Interpolationskoeffizienten dar, welche errechnet werden müssen. Diese Koeffizienten werden durch die Bedingung definiert, dass $s(x_i)\overset{!}{=}f_i~\forall i$ gelten muss. Aus dieser Bedingung erhält man ein zu lösendes Gleichungssystem, welches in der folgenden Gleichung dargestellt ist.
\begin{align}
\underbrace{\begin{bmatrix}
 \phi_1(x_1) & \cdots & \phi_n(x_1) \\ \vdots & \ddots & \vdots \\ \phi_1(x_n) & \cdots & \phi_n(x_n)
\end{bmatrix}}_{A}
\underbrace{\begin{bmatrix}
 \lambda_1 \\ \vdots \\ \lambda_n
\end{bmatrix}}_{\lambda}
=
\underbrace{\begin{bmatrix}
 f_1 \\ \vdots \\ f_n
\end{bmatrix}}_{f}\label{eq:Intermatrix}
\end{align}
Im eindimensioanlen Fall gilt für viele Basisfunktionen, dass die Basisfunktionsmatrix regulär ist, wenn die einzelnen Knotenpunkte $\{x_i\}_{i=1}^n$ disjunkt zueinander sind. Daraus folgt die eindeutige Lösbarkeit des Gleichungssystems und somit die Eindeutigkeit der Koeffenzienten $\lambda_i$. Im mehrdimensionalen Fall kann die Koeffenzientenmatrix ihre Regularitätseigenschaft verlieren. Der Beweis der Nicht-Regularität liefert das \textit{Haar-Mairhuber-Curtis-Theorem}, welches in \cite{de2013four} beschrieben und bewiesen wird, sodass es hier nicht näher erläutert wird.\\
Um die Regularität auch im Mehrdimensionialen zu gewährleisten, nutzt man die sogenannten radialen Basisfunktionen (RBF). Genauer wird, anstelle einer Linearkombination von einer Menge von Basisfunktion, welche unabhängig von ihren Stützstellen sind, eine Linearkombination von Translationen einer festvorgegebenen RBF verwendet. Diese RBF haben die Eigenschaften, dass sie (radial) symmetrisch zu ihren Zentren sind, welches bedeutet, dass sie invariant bezüglichen Rotationen sind. Das heißt, eine Funktion $\psi:\mathbb{R}^n\rightarrow\mathbb{R}$ ist genau dann eine RBF, wenn sie die Bedingung $\psi(x)=\psi(\|x\|)~\forall x\in\mathbb{R}$ erfüllt. Einige Beispiele für unendlich glatte, radiale Basisfunktionen sind in Tabelle \ref{tbl:funktionstabelle} wiederzufinden.
\begin{table}[H]
\centering
\begin{tabular}{|l|l|}
\hline
\textbf{unendlich glatte RBF}  & \textbf{$\psi(x), k>0$} \\
\hline
\rule{0pt}{15pt}Gauß-Funktion&~~~~~$e^{-(kx)^2}$\\
\rule{0pt}{15pt}Multiquadratische-Funktion& $\sqrt{1+(kx)^2}$\\
\rule{0pt}{15pt}Invers-Multiquadratische-Funktion&~~$\frac{1}{\sqrt{1+(kx)^2}}$\\
\hline
\end{tabular}
\caption{Beispieltabelle für radiale Basisfunktionen}
\label{tbl:funktionstabelle}
\end{table}
\noindent
Der freiwählbare Faktor $k$ stellt in den Funktionen den Glättungsparameter dar. Zur besseren Veranschaulichung der radialen Symmetrie ist die Gauß-Funktion und die Multiquadratische-Funktion in Abbildung \ref{fig:RBF_1dim} im eindimensionalen und in Abbildung \ref{fig:RBF_2dim} im zweidimensionalen Fall dargestellt.
\begin{figure}[H]
	\centering
\hspace{-1.4cm}
\begin{minipage}{0.45\textwidth}
	\includegraphics[width=1.22\textwidth]{figures/1_dim_gauss.png}
	\caption*{(i)Gauß-Funktion}
\end{minipage}
\hspace{0.5cm}
\begin{minipage}{0.45\textwidth}
	\includegraphics[width=1.22\textwidth]{figures/1_dim_multi.png}
	\caption*{(ii)Multiquadratische-Funktion}
\end{minipage}
	\caption{1-dimensionale radiale Basisfunktionen}
	\label{fig:RBF_1dim}
\end{figure}
\begin{figure}[H]
	\centering
\hspace{-1.5cm}
\begin{minipage}{0.45\textwidth}
	\includegraphics[width=1.25\textwidth]{figures/2_dim_gauss.png}
	\caption*{(i)Gauß-Funktion}
\end{minipage}
\hspace{0.6cm}
\begin{minipage}{0.45\textwidth}
	\includegraphics[width=1.25\textwidth]{figures/2_dim_multi.png}
	\caption*{(ii)Multiquadratische-Funktion}
\end{minipage}
	\caption{2-dimensionale radiale Basisfunktionen}
	\label{fig:RBF_2dim}
\end{figure}
\noindent
Nun erhält man nach einer Wahl der RBF die neue Interpolationsfunktion
\begin{align}
 s(x)=\sum_{i=1}^n\lambda_i\psi(\|x-x_i\|).
\end{align}
Es ist zu beachten, dass die Regularität der Koeffizientenmatrix nicht für alle RBF gewährleistet ist. Um diese Eigenschaft für alle RBF zu erhalten, wird die Interpolierende mit einer Linearkombination von Polynomen erweitert:
\begin{align}
 s(x)=\sum_{i=1}^n\lambda_i\psi(\|x-x_i\|)+\sum_{j=1}^M\gamma_jp_j(x)
\end{align}
Hierbei stellen die Polynome $\{p_k\}_{k=1}^M$ eine Basis des Raums $\mathbb{P}_m^d$ und $M=dim\{\mathbb{P}_m^d\}$ die obere Grenze der zweiten Summe dar. Dabei repräsentiert $\mathbb{P}_m^d$ den Raum aller d-variaten Polynome vom Grad kleiner als m.\col{variate Polynome?} Durch diese Erweiterung treten nun weitere Freiheitsgrade auf, welche durch die folgende Bedingung definiert werden:
\begin{align}
 \sum_{j=1}^n\lambda_jp_k(x_j)~k=1,\dots,M
\end{align}
Somit kann das Gleichungssystem aus Gleichung \eqref{eq:Intermatrix} wie folgt erweitert werden:
\begin{align}
{\large
\underbrace{\begin{bmatrix}
 A & P \\ P^T & 0
\end{bmatrix}}_{\tilde A}
\underbrace{\begin{bmatrix}
 \lambda  \\ \gamma
\end{bmatrix}}_{\tilde\lambda}
=
\underbrace{\begin{bmatrix}
 f  \\ 0
\end{bmatrix}}_{\tilde f}\label{eq:Inter}}
\end{align}
In diesem Zusammenhang sind die Matrizen $A\in\mathbb{R}^{n\times n}$ und $P\in\mathbb{R}^{n\times M}$. Die Elemente der Matrix $A$ und $P$ lauten $a_{ij}=\psi(\|x_i-x_j\|)$ und $p_{kl}=p_k(x_l)$ für $1\le i,j,l\le n$ und $1\le k\le M$. Der Beweis der Regularität für die erweiterte Koeffizientenmatrix $\tilde A$ liefert C.A.Micchellini in seiner Ausarbeitung~\cite{micchelli1984interpolation}, sodass dieses speziell gewählte Gleichungssystem eine eindeutige Lösung liefert.
\subsection{Das RBF-FD-Verfahren}
In diesem Kapitel werden die Erkenntnisse der letzten beiden Kapitel zusammengebracht, um ein Verfahren zu erschaffen, welches die Simplizität des FD-Verfahrens und die Gitter-Unabhängigkeit der RBF-Interpolation verbindet und somit das RBF-FD-Verfahren formt. Die Aussagen dieses Kapitels gehen auf die Arbeit von Bengt Fornberg und Natasha Flyer~\cite{fornberg2015primer} zurück.\\
Das RBF-FD-Verfahren ist ein relativ junges Verfahren zur Approximation von Differentialgleichungen, welches ihre Anfänge erst in den 2000er Jahren hatte. Ein Vertreter der ersten Publizierungen war die Ausarbeitung~\cite{tolstykh2003using} von  A.I.Tolstykh und D.A.Shirobokov im Jahre 2003.\\
Der Grundgedanke des RBF-FD-Verfahrens ist eine Kombination der Finiten-Differenzenmethode und der RBF-Interpolation, welches auf einem Gebiet mit zufällig ausgewählten, disjunkten Auswertungsknoten eine Differentialgleichung approximiert. Sei nun $\overline{\Omega}=\Omega\cup\partial\Omega$ das zugrunde liegende Gebiet zu der allgemeinen Problemstellung:
\begin{align}
 L\{u(x)\}=f(x),~x\in\overline{\Omega}\label{eq:FD-allgemein}
\end{align}
Hierbei stellen $u:\overline{\Omega}\rightarrow\mathbb{R}$ die zu approximierende Lösung, $L$ einen linearen Differentialoperator und $f$ eine vordefinierte Funktion dar. Wendet man nun eine Diskretisierung des Gebietes an, bei der lediglich einzelne Punkte gewählt werden, so erhält man eine gitterfreie Diskretisierung mit 
\begin{align*}
 \overline{\Omega}_h=\{x_i\in\overline{\Omega}|i\in I\text{ und }x_i\neq x_j,\forall i\neq j\}\subset\overline{\Omega},
\end{align*}
welches die disjunkten Auswertungsknoten beinhaltet. Hierbei stellt $I=\{1,\dots,N\}$ mit $N\in\mathbb{N}$ eine Indexmenge dar. Da im späteren Verlauf des Verfahrens diese diskreten Punkte $x_i$ die Zentren der radialen Baissfunktionen darstellen, wird die Menge $\overline{\Omega}_h$ neu definiert
\begin{align*}
 \overline{\mathcal{Z}}=\{z_i|i\in I\text{ und }z_i\neq z_j,\forall i\neq j\}:=\overline{\Omega}_h
\end{align*}
und nun als Menge aller Zentren bezeichnet.\\
Sei nun $L_h$ der diskrete Differentialoperator, welcher mittels einem Finiten-Differenzen-Ansatz, wie im Kapitel \ref{sec:FD} erläutert wurde, approximiert wird. Somit kann \eqref{eq:FD-allgemein} an einem Auswertungsknoten $z_i$ wie folgt diskretisiert werden:
\begin{align}
 L_h\{u(z_i)\}=\sum_{j\in J}w_{ij}u(z_j)=f(z_i)~\forall i\label{eq:rbffd-1}
\end{align}
Dabei stellt $J=\{i,i_1,\dots,i_M\}\subset I$ mit $M\in\mathbb{N}_0$ die Indexmenge der von dem Zentrum $z_i$ induzierten Stern-Menge und $w_{ij}\in\mathbb{R}$ die unbekannten FD-Gewichte dar. Wegen der zufälligen Positionierung der Auswertungsknoten sind diese Gewichte von vornherein nicht bekannt. Somit ist die Hauptaufgabe des RBF-FD-Verfahrens die Berechnung dieser Gewichte. Bei der Bestimmung dieser Gewichte wird nun die Funktion $u$ durch eine Interpolierende ersetzt. Hier kommt nun die in Kapitel \ref{sec:rbf_inter} vorgestellte RBF-Interpolation zum Einsatz. Sei nun
\begin{align}
 s(x)=\sum_{k=1}^N\lambda_k\psi(\|x-x_k\|)\approx u(x)
\end{align}
die Interpolierende zu der Funktion $u$. Dabei seien $\lambda_k\in\mathbb{R}$ die Interpolationskoeffizienten und $\psi(\cdot)$ die radialen Basisfunktionen, welche mit einer Norm $\|\cdot\|$ definiert sind. Setzt man nun die Interpolierende in die Gleichung \eqref{eq:rbffd-1} ein und definiert übersichtshalber $\zeta:=z_i$, so erhält man die folgenden Äquivalenzumformungen:
\begin{align}
  L_h\{u(\zeta)\}&=\sum_{j\in J}w_{ij}u(z_j)\\
\Leftrightarrow L_h\{\sum_{k=1}^N\lambda_i\psi(\|\zeta-z_k\|)\}&=\sum_{j\in J}w_{ij}\sum_{k=1}^N\lambda_k\psi(\|z_j-z_k\|)\\
\Leftrightarrow \sum_{k=1}^N\lambda_iL_h\{\psi(\|\zeta-z_k\|)\}&=\sum_{k=1}^N\lambda_k\sum_{j\in J}w_{ij}\psi(\|z_j-z_k\|)\\
\Leftrightarrow~~~~~~~ \underbrace{L_h\{\psi(\|\zeta-z_k\|)\}}_{R_i(L_h)}&=\sum_{j\in J}\underbrace{w_{ij}}_{w_i}\underbrace{\psi(\|z_j-z_k\|)}_{\Psi_i}
\end{align}
Aus der letzten Äquivalenzumformung geht hervor, dass die Gewichte $w_{ij}$ zur Berechnung der Lösung am Knoten $z_i$ durch das lineare Gleichungssystem
\begin{align}
 \Psi_i w_i=R_i(L_h)\label{eq:lgs-distanz}
\end{align}
definiert werden. Hierbei stellt $\Psi_i\in\mathbb{R}^{M\times M}$ die Distanzmatrix und $R_i(L_h)$ die rechte Seite des Gleichungssystems dar, welches abhängig von dem diskreten Differentialoperator ist. Das lineare Gleichungssystem in  Gleichung \eqref{eq:lgs-distanz} hat die Form:
\begin{align*}
 {\footnotesize
\underbrace{\begin{bmatrix}
  \psi(\|z_1-z_1\|) & \psi(\|z_2-z_1\|) & \cdots & \psi(\|z_M-z_1\|) \\
 \psi(\|z_1-z_2\|) & \psi(\|z_2-z_2\|) & \cdots & \psi(\|z_M-z_2\|) \\
 \vdots & \vdots & \ddots & \vdots \\
 \psi(\|z_1-z_M\|) & \psi(\|z_2-z_M\|) & \cdots & \psi(\|z_M-z_M\|)
\end{bmatrix}}_{\Psi_i}
\underbrace{\begin{bmatrix}
 w_{i1} \\
 w_{i2} \\
 \vdots \\
 w_{iM}
\end{bmatrix}}_{w_i}
=
\underbrace{\begin{bmatrix}
 L_h\{\psi(\|\zeta-z_1\|)\} \\
 L_h\{\psi(\|\zeta-z_2\|)\} \\
 \vdots \\
 L_h\{\psi(\|\zeta-z_M\|)\}
\end{bmatrix}}_{R_i(L_h)}
}
\end{align*}
Nach der Berechnung der FD-Gewichte können die einzelnen Werte an den Auswertungsknoten, wie in einem klassischen FD-Verfahren, ermittelt werden.
\subsection{RBF-FD angewendet auf die Transportgleichung}
Um nun die im vorherigen Kapitel beschriebenen, theoretischen Überlegungen anzuwenden, wird die RBF-FD-Methode in dieser Arbeit auf die inkompressible Transportgleichung der Form \eqref{eq:transport3} angewendet. Hierzu wird die Metrik
\begin{align}
 s:=\frac{\|x-y\|_2^2}{2},
\end{align}
welche zur Berechnung der Distanzen zwischen den einzelnen Knoten eingesetzt wird, verwendet. Dabei stellen $x,y\in\mathbb{R}^d$  Vektoren und $\|\cdot\|_2$ die euklidische Norm dar. Somit folgt, dass der Differentialoperator $L\{\cdot\}$ aus dem vorherigen Kapitel die Form
\begin{align}
 L\{u\}=\textbf{v}\cdot\nabla u
\end{align}
besitzt. Demnach kann das Gleichungssystem aus \eqref{eq:lgs-distanz} geschrieben werden als:
\begin{align*}
\renewcommand\arraystretch{1.57}
{\footnotesize
\begin{bmatrix}
  \psi\left(\frac{\|z_1-z_1\|_2^2}{2}\right) & \psi\left(\frac{\|z_2-z_1\|_2^2}{2}\right) & \cdots & \psi\left(\frac{\|z_M-z_1\|_2^2}{2}\right) \\
 \psi\left(\frac{\|z_1-z_2\|_2^2}{2}\right) & \psi\left(\frac{\|z_2-z_2\|_2^2}{2}\right) & \cdots & \psi\left(\frac{\|z_M-z_2\|_2^2}{2}\right) \\
 \vdots & \vdots & \ddots & \vdots \\
 \psi\left(\frac{\|z_1-z_M\|_2^2}{2}\right) & \psi\left(\frac{\|z_2-z_M\|_2^2}{2}\right) & \cdots & \psi\left(\frac{\|z_M-z_M\|_2^2}{2}\right)
\end{bmatrix}
\begin{bmatrix}
 w_{i1} \\
 w_{i2} \\
 \vdots \\
 w_{iM}
\end{bmatrix}
=
\begin{bmatrix}
 \textbf{v}\cdot\nabla\psi\left(\frac{\|\zeta-z_1\|_2^2}{2}\right) \\
 \textbf{v}\cdot\nabla\psi\left(\frac{\|\zeta-z_2\|_2^2}{2}\right) \\
 \vdots \\
 \textbf{v}\cdot\nabla\psi\left(\frac{\|\zeta-z_M\|_2^2}{2}\right)
\end{bmatrix}
}
\end{align*}
Um dieses Gleichungssystem lösen zu können, wird im Folgendem ein genauerer Blick auf die Berechnung der rechten Seite, beziehungsweise auf die Berechnung der Ableitungen von den radialen Basisfunktionen, geworfen. 
Die skalaren Ableitungen der Metrik $s$ haben die Form:
\begin{align}
 \frac{\partial s}{\partial x_i}&=\frac{2(x_i-y_i)}{2}=x_i-y_i\\
 \frac{\partial s}{\partial y_i}&=\frac{-2(x_i-y_i)}{2}=y_i-x_i
\end{align}
Somit gilt unter Anwendung der Kettenregel für die Ableitungen für radiale Basisfunktionen mit dieser Metrik:
\begin{align}
  \frac{\partial}{\partial x_i}\psi(s)&= \psi'(s)\frac{\partial s}{\partial x_i}=\psi'(s)(x_i-y_i)\\
 \frac{\partial}{\partial y_i}\psi(s)&= \psi'(s)\frac{\partial s}{\partial y_i}=\psi'(s)(y_i-x_i)
\end{align}
Mit diesen Ableitungen lautet die i-te Zeile der rechten Seite des Gleichungssystems
\begin{align}
\textbf{v}\cdot\nabla\psi\left(\frac{\|\zeta-z_i\|_2^2}{2}\right)
 =
\textbf{v}\cdot
\begin{bmatrix}
\frac{\partial}{\partial x_i}\psi\left(\frac{\|\zeta-z_i\|_2^2}{2}\right)\\
\frac{\partial}{\partial y_i}\psi\left(\frac{\|\zeta-z_i\|_2^2}{2}\right) 
\end{bmatrix}
=
\textbf{v}\cdot
\begin{bmatrix}
\psi'\left(\frac{\|\zeta-z_i\|_2^2}{2}\right)(x_i-y_i) \\
\psi'\left(\frac{\|\zeta-z_i\|_2^2}{2}\right)(y_i-x_i)
\end{bmatrix}\nonumber
\end{align}
mit $z_i=(x_i,y_i)^T$.\\
Nach der Berechnung der Gewichte, kann nun die Differentialgleichung durch die Finite-Differenzen-Methode approximiert werden. Somit kann die Gleichung \eqref{eq:transport3} in den algebraischen Fall umgeschrieben werden zu
\begin{align}
 M\frac{\partial u}{\partial t}=K^Hu\label{eq:algTrans},
\end{align}
wobei $M$ die Einheitsmatrix dar und $K^H$ die Steifigkeitsmatrix aus der RBF-FD-Methode dar.
\pagebreak
\section{Flux Correction Tools - FCT}\label{sec:FCT}
Dieses Kapitel bezieht sich hauptsächlich auf den Artikel~\cite{kuzmin2002flux} und erläutert ein Vorgehen zur Stabilisierung von numerischen Approximationsverfahren von partiellen Differentialgleichungen.\\
Viele Anwendungsbereiche für partielle Differentialgleichungen und somit auch für die \textit{Transportgleichung} findet man in dem Bereich der Physik wieder. Diesbezüglich muss die Lösung dieser Gleichung aus physikalischen Gründen positiv sein. Somit benötigt man ein Verfahren, welches die Positivitätseigenschaft beibehält. Es gibt sehr viele verschiedene Verfahren, welche ihre Vor- und Nachteile diesbezüglich besitzen. Klassische-Upwind-Verfahren, welche zur Berechnung der Lösung Knotenwerte aus der Stromaufwärtsrichtung nehmen, erfüllen diese Positivitätsbedingung, weisen aber auch diffusive Merkamele auf, welches nicht für alle numerischen Probleme erwünscht wird~\cite{kuzmin2002flux}. Im Gegensatz dazu sind die Verfahren höherer Ordnung(High-order-Methoden) nicht diffusiv, aber weisen vermehrt oszillierende Ergebnisse auf~\cite{kuzmin2002flux}. Das FCT-Schema verbindet nun Verfahren, welche diese oben genannten Eigenschaften aufweisen, und bekommt so eine nicht-diffusive, nicht-oszillierende Lösung heraus. Somit ist die Aufgabe des Schemas die Stabilisation der errechneten Lösung. Allgemein werden drei Hauptschritte druchgeführt:
\begin{enumerate}
 \item Diskretisierung mittels einer High-order-Methode
 \item Addition einer künstlichen Diffusion, um die Positivitätseigenschaft zu gewährleisten
 \item Limitierte Subtraktion der künstlichen Diffusion, um wieder eine high-order-Lösung zu bekommen
\end{enumerate}
Der erste Schritt kann durch unterschiedliche High-order-Methoden durchgeführt werden. Beispiele hierfür wären die \textit{Finite-Elemente-Methode} oder der \textit{RBF-FD-Methode}, welche im Kapitel \ref{sec:RBF-FD} vorgestellt wurde. Im Folgenden werden die letzten beiden Schritte anhand der in Kapitel \ref{sec:RBF-FD} aufgeführten \textit{algebraischen Transportgleichung} \eqref{eq:algTrans} genauer erläutert. Zur Erinnerung, $M$ ist die Massenmatrix, $K^H$ die Matrix, welche den diskreten Transport-Operator darstellt und $u$ die Lösung des Verfahrens höherer Ordnung. Wie oben schon erwähnt, muss die Lösung $u$ die Positivitätseigenschaften erfüllen. Um dies in jedem Zeitschritt zu gewährleisten, diskretisieren wir die Zeitdifferentation mittels dem allgemeinen \textit{Theta-Schema}:
\begin{align}
 \frac{u^{n+1}-u^n}{\Delta t}=(1-\Theta)F(t_n,u^n)+\Theta F(t_{n+1},u^{n+1})~~~,0\le\Theta\le1
\end{align}
Hierbei bezeichnet $F(t_n,u^n)$ die rechte Seite der Differentialgleichung zum Zeitpunkt $t_n$. Im Fall der Transportgleichung ist $F(t_n,u^n)=K^Hu^n$, sodass nun die Gleichung \eqref{eq:algTrans} geschrieben werden kann als:
\begin{align}
 M(\frac{u^{n+1}-u^n}{\Delta t})&=(1-\Theta)K^Hu^n+\Theta K^Hu^{n+1}\nonumber\\
\Rightarrow [M-\Theta\Delta tK^H]u^{n+1}&=[M+(1-\Theta)\Delta tK^H]u^n\label{eq:thetaGleich}
\end{align}
Hierbei stellen $u^{n+1}$ und $u^n$ jeweils die Lösung $u$ im Zeitschritt $n+1$  und $n$ dar. Das \textit{Theta-Schema} gehört zu den Einschrittverfahren und verbindet durch den Paramter $\Theta$ explizite und implizite Methoden. Somit definieren die Fälle
\begin{enumerate}
 \item $\theta=0$: das \textit{explizite Eulerverfahren}
 \begin{enumerate}
  \item[$\Rightarrow$] $Mu^{n+1}=[M+\Delta tK]u^n$
 \end{enumerate}
 \item $\theta=1$:    das \textit{imlizite Eulerverfahren}
 \begin{enumerate}
  \item[$\Rightarrow$] $[M-\Delta tK^H]u^{n+1}=Mu^n$
 \end{enumerate}
 \item $\theta=\frac{1}{2}$:  das \textit{Crank-Nicolson-Verfahren}
 \begin{enumerate}
  \item[$\Rightarrow$] $[M-\frac{1}{2}\Delta tK^H]u^{n+1}=[M+\frac{1}{2}\Delta tK^H]u^n$
 \end{enumerate}
\end{enumerate}
drei Spezialfälle von Einschrittverfahren.\\
Im Folgendem wird der Beweis der Positivität der Lösung $u$ für den Fall des \textit{imlizite Eulerverfahrens}
\begin{align}
 [M-\Delta tK^H]u^{n+1}=Mu^n\label{eq:LGS_H}
\end{align}
vorgestellt. Für die Verfahren für $0\le\Theta<1$ wird auf den Beweisen des zugrundeliegenden Artikels~\cite{kuzmin2002flux} verwiesen.\\
Sei nun angenommen, dass die Lösung des vorherigen Zeitschrittes $u^n\ge0$ ist und dass die Massenmatrix $M$ eine Diagonalmatrix mit $m_{ii}>0~\forall i$ ist, so folgt die Positivitätseigenschaft für $u^{n+1}$, wenn $[M-\Delta tK^H]^{-1}>0$ ist. Diese Aussage ist äquivalent zu der Definition einer \textit{M-Matrix}.
\begin{defi}\label{def:MMatrix}
 \textbf{(M-Matrix~\cite{fiedler2008special}):}\\
Eine reguläre Matrix $A\in \mathbb{R}^{n\times n}$ heißt M-Matrix, falls $A$ folgende Eigenschaften erfüllt:
\begin{enumerate}
 \item $A$ ist eine Z-Matrix, d.h. $a_{ij}\le0~\forall i\neq j$
 \item $\exists v\in\mathbb{R}^n\text{ mit }v_i>0~\forall i:(Av)_i>0~\forall i$
\end{enumerate}
Sind diese Kriterien erfüllt gilt
\begin{align*}
 A^{-1}>0.
\end{align*}
\end{defi}
 Die Matrix $[M-\Delta tK^H]$ muss somit so modifiziert werden, dass alle Nebendiagonalelemente kleiner gleich Null sind. Hierbei ist zu beachten, dass die Nebendiagonalelemente nur von $K^H$ abhängen, da $M$ eine Diagonalmatrix ist. Somit addieren wir nun zu der Transport-Operator-Matrix eine Matrix so hinzu, dass sichergestellt werden kann, dass alle Nichtdiagonalelemente positiv werden:
\begin{align}
 K^L=K^H+D\label{eq:AddDif}
\end{align}
Physikalisch betrachtet, fügen wir mit der Matrix $D$ eine künstliche Diffusion hinzu, sodass die Lösung $u$ nun diffusivere Merkmale aufweisen könnte~\cite{kuzmin2002flux}. Somit bezeichnet man diese Lösung als $u_{Low}$ bzw. $u_L$.\\
Weiterhin ist zu beachten, dass eine Addition mit zu starker Diffusion die Folge einer Absenkung der Stabilitätsgrenzen hat~\cite{kuzmin2002flux}. Andererseits führt eine zu schwache Diffusionsaddition zu falschen Extrema. Somit könnte man überall die gleiche Menge an Diffuion hinzuaddieren. Diese Methode ist zwar recheneffizient, aber nicht genauigkeitseffizient~\cite{kuzmin2002flux}. Um die obengenannten Eigenschaften nicht zu erhalten, werden die einzelnen Matrixeinträge für jeden Knoten wie folgt definiert:
\begin{align}
 d_{ii}=-\sum_{k\neq i}d_{ik},~~d_{ij}=d_{ji}=max\{0,-k_{ij}^H,-k_{ji}^H\},~~\forall i<j\label{eq:D}
\end{align}
Bei dieser Definition fällt auf, dass die Zeilen- und Spaltensumme von $D$ gleich Null sind. Somit gilt insgesamt für die Zeilensumme von $K^L$, dass diese gleich Null ist. Zu beachten ist hier, dass, wenn die nichtdiagonal Elemente von $K^H$ von vornherein positiv sind, keine Addition künstlicher Diffusion notwendig ist und $D$ gleich der Nullmatrix ist. So folgt $K^H=K^L$. Dieser Fall tritt auf, wenn die physikalische Diffusion von Anfang an stark genug ist.\\
Setzt man nun die Matrix $K^L$ in die Gleichung \eqref{eq:LGS_H} ein, folgt:
\begin{align}
 [M-\Delta tK^L]u_L^{n+1}=Mu^n
\end{align}
Substituiere nun $\widehat{K}:=[M-\Delta tK^L]$. Durch die Addition der künstlichen Diffusion gilt nun $\widehat{k}_{ij}=-\Delta tk_{ij}^L\le0~\forall i\neq j$. Durch die künstliche Diffusion sind somit auch die Nebendiagonalelemente von $\widehat{K}$ negativ und erfüllt somit Punkt 1 der Definition \ref{def:MMatrix}.\\
Um den Punkt 2 zu beweisen, zeigen wir zunächst, dass die Zeilensumme von $K^L$ gleich Null ist. Diese Eigenschaft ergibt sich aus den Zeilensummeneigenschaften von $K^H$ und $D$. Durch die Annahme eines inkompressiblen Vektorfeldes gilt für die Koeffenzientenmatrix $K^H$ die Zeilensummeneigenschaften $(K^Hu)_i=\sum_{j}k_{ij}^Hu_j=0$~\cite{kuzmin2005algebraic}. Diese Eigenschaft gilt auch für die Matrix $D$, welches man anhand der Definition in Gleichung~\eqref{eq:D} erkennen kann. Somit gilt nun für die Matrix $K^L$:
\begin{align}
 (K^Lu)_i=\sum_{j}k_{ij}^Lu_j=[\underbrace{\sum_{j}k_{ij}^H+d_{ij}}_{=0}]u_j=0
\end{align}
Wähle nun den Vektor $v=(1,\dots,1)^T>0$. Somit folgt mit der oben gezeigten Zeilsummeneigenschaft von $K^L$:
\begin{align}
 \sum_j \widehat{k}_{ij} v_j=m_{ii} - \Delta t\underbrace{\sum_j k_{ij}^L}_{=0}=m_{ii}>0~~\forall i
\end{align}
Die Regularität der Matrix $\widehat{K}$ folgt aus den Überlegungen aus Kapitel \ref{sec:RBF-FD}. Somit folgt die Eigenschaft der \textit{M-Matrix}, welche die Positivitätseigenschaft für $u^{n+1}$ impliziert.\\\\
Das Vorgehen der Addition einer künstlichen Diffusion ist auch in den Fällen für $\theta\in[0,1)$ anwendbar. Somit erhalten wir mit den gleichen Überlegungen wie im Beweis der Positivität der Lösung für das \textit{implizite Eulerverfahren} die Gleichung
\begin{align}
 [M-\Theta\Delta t\underbrace{K^L}_{\scriptscriptstyle=K^H+D}]u^{n+1}&=[M+(1-\Theta)\Delta t\underbrace{K^L}_{\scriptscriptstyle=K^H+D}]u^n.
\end{align}
Als nächsten Schritt muss die Addition der künstlichen Diffusion wieder ausgeglichen werden. Hierzu wird der Additionsschritt auf der rechten Seite wieder abgezogen und man erhält die Gleichung
\begin{align}
 [M-\Theta\Delta tK^L]u^{n+1}&=u^{n+\Theta} -(1-\theta)\Delta tDu^n-\Theta\Delta tDu^{n+1}\nonumber\\
 &=u^{n+\Theta}-\Delta tf(u^n,u^{n+1})\label{eq:zwischGleich}
\end{align}
mit den Substitutionen
\begin{align}
 u^{n+\Theta}=[M+(1-\Theta)\Delta tK^L]u^n \text{ und } f(u^n,u^{n+1})=(1-\Theta)Du^n+\Theta Du^{n+1}.\nonumber
\end{align}
Die subtrahierte Diffusion $f(u^n,u^{n+1})$ wird im Folgendem als Antidiffusion bezeichnet. Der Eintrag $f_i$ des Antidiffuionsvektor kann als Summe der Flüsse, welche den Wert am Knoten $i$ beeinflussen, aufgefassen werden:
\begin{align}
 f_i=\sum\limits_{j\neq i}f_{ij}\text{ mit }f_{ij}=(1-\Theta)d_{ij}(u_i^n-u_j^n)+\Theta d_{ij}(u_i^{n+1}-u_j^{n+1})=-f_{ji}\nonumber
\end{align}
Durch die geeignete Wahl der Antidiffusion ist nun die Gleichung~\eqref{eq:zwischGleich} äquivalent zu der Gleichung~\eqref{eq:thetaGleich} und besitzt somit die gleiche analytische Lösung. Hierbei wird nun im Gegensatz zu Schritt 2 für jeden Knoten einzeln die Stärke dieser Antidiffusion limitiert, sodass keine zu hohe bzw. zu niedrige Diffusionskompensation durchgeführt wird. Somit folgt mit den Diffusionskoeffizientenvektor $\alpha$ die limitierte Gleichung
\begin{align}
  [M-\Theta\Delta tK^L]u^{n+1}=u^{n+\Theta}-\Delta t\alpha f(u^n,u^{n+1}).\label{eq:alphaGleich}
\end{align}
Im Allgemeinen besitzen die Koeffizienten von $\alpha$ Werte im Intervall $[0,1]$. Wenn beispielsweise $\alpha\approx0$ ist, dann muss die künstliche Diffusion nicht ausgeglichen werden und es gilt:
\begin{align}
 [M-\Theta\Delta tK^L]u^{n+1}&=[M+(1-\Theta)\Delta tK^L]u^n
\end{align}
Im Gegensatz dazu gilt für $\alpha\approx1$, dass $u^{n+1}$ gleich der anfänglichen Lösung der Gleichung~\eqref{eq:thetaGleich} ist.\\
Im Folgenden wird die genaue Herleitung der Koeffizienten von $\alpha$ erläutert. Hierzu werden die Überlegungen von Zalesaks multidimensionalen FCT-Algorithmus verwendet\cite{zalesak1979fully}. Man betrachte die Gleichung~\eqref{eq:alphaGleich} für einen Knoten $i$. Dann kann die rechte Seite umgeschrieben werden in:
\begin{align}
 RHS=m_{ii}u_i^{n+\Theta}+\Delta t\sum\limits_{j\neq i}\alpha_{ij}f_{ij},~~~\alpha_{ij}=\alpha_{ji}
\end{align}
Um die Positivitätseigenschaft der limitierten Lösung zu gewährleisten muss die Lösung $u^{n+\Theta}$ positiv sein und die Diffusionskoeffizienten entsprechend gewählt werden. Die Positivitätseigenschaft der Zwischenlösung wird im zugrundeliegenden Artikel~\cite{kuzmin2002flux} bewiesen. Somit hängt die Positivität der Gesamtlösung von der expliziten Wahl der Diffusionskoeffizienten $\alpha_{ij}$ ab. In den Überlegungen von Zalesak wird ein 4 Schritt-Algorithmus zur Berechnung der Diffusionskoeffizienten verwendet, welcher im Folgenden vorgestellt wird:
\begin{enumerate}
 \item Berechnung der Flüsse im Knoten $i$ und Aufteilung in positiv und negativ:
\begin{align}
P_i^+=\frac{1}{m_{ii}}\sum\limits_{j\neq i}max\{0,f_{ij}\},~~~~P_i^-=\frac{1}{m_{ii}}\sum\limits_{j\neq i}min\{0,f_{ij}\}
\end{align}
 \item Berechnung der maximalen Distanz zum lokalen Extremum um den Knoten $i$:
 \begin{align}
Q_i^+&=max\left\{0,\smash{\displaystyle\max_{j\neq i}}(u_j^{n+\Theta}-u_i^{n+\Theta})\right\}\nonumber\\
Q_i^-&=min\left\{0,\smash{\displaystyle\min_{j\neq i}}(u_j^{n+\Theta}-u_i^{n+\Theta})\right\}
\end{align}
 \item Berechnung des Knotenfaktors, um die Positivität zu gewährleisten:
\begin{align}
R_i^+=min\left\{1,\frac{Q_i^+}{\Delta tP_i^+}\right\},~~~~R_i^-=min\left\{1,\frac{Q_i^+}{\Delta tP_i^+}\right\}
\end{align}
 \item Limitierung der Flüssen mittels den Minimum aus $R_i^+$ und $R_i^-$:
\begin{align}
\alpha_{ij}=   \begin{cases}
     min\{R_i^+,R_j^-\}, & \text{wenn } f_{ij}>0 \\
     min\{R_i^-,R_j^+\}, & \text{sonst} 
   \end{cases}
\end{align}
\end{enumerate}
Mit dieser Definition der Diffusionskoeffizienten folgt die Beschränktheit der rechten Seite, denn es gilt:
\begin{align}
 min\{u_i^{n+\Theta}\}=u_i^{n+\Theta}+Q_i^-\le\frac{RHS}{m_{ii}}\le u_i^{n+\Theta}+Q_i^+=max\{u_i^{n+\Theta}\}
\end{align}
Des Weiteren verstärkt die limitierte Antidiffusion nicht lokale Extrema, da in diesem Fall
\begin{align}
 Q_i^\pm=0~\Rightarrow~ R_i^\pm=0~\Rightarrow~\alpha_{ij}=0
\end{align}
gilt und somit die limitierte Antidiffuion wegfällt.\\
Abschließend kann man das FCT-Schema durch folgenden Algorithmus zusammenfassen:
\begin{enumerate}
 \item Berechnung der Lösung $u^{n+1}$ aus dem Verfahren höherer Ordnung
 \item Berechnung der Zwischenlösung $u^{n+\Theta}=[M+(1-\Theta)\Delta tK^L]u^n$
 \item Berechnung von $f(u^n,u^{n+1})$ und der Koeffizienten von $\alpha$
 \item Berechnung der stabilisierten High-order-Lösung $u_H^{n+1}$ mit der Gleichung $[M-\Theta\Delta tK^L]u_H^{n+1}=u^{n+\Theta}+\Delta t\alpha f(u^n,u^{n+1})$
\end{enumerate}
Zusammenfassend bekommt man durch das Anwenden des FCT-Schema eine stabilisierte Lösung heraus, als das reine high-order-Verfahren. Die FCT-Methoden sind allgemein auf explizite und implizite Zeitdiskretisierungen anwendbar und sind nicht auf die Raumdimension beschränkt~\cite{kuzmin2002flux}.\\ Numerische Beispiele zur praktischen Analyse werden im folgenden Kapitel~\ref{sec:NumVers} vorgestellt. 
\pagebreak
\section{Numerische Versuche}\label{sec:NumVers}
In diesem Kapitel wird das RBF-FD-Verfahren und die FCT-Stabilisation, welche in den vorherigen Kapiteln vorgestellt wurden, anhand numerischer Beispiele graphisch erläutert. Hierzu werden numerische Fehleranalysen der verschiedenen Lösungen betrachtet und am Ende ein Fazit gezogen. Für die Auswertungen wurde eine Matlab-Implementierung der Version \textit{R2016a} von Herrn Dr. Sokolov \& Co. verwendet.
\subsection{Numerische Resultate der RBF-FD-Methode}
\subsection{Numerische Resultate der FCT-Stabilisation}\label{sec:num_res_FCT}
In diesem Unterabschnitt wird ein genauerer Blick auf die FCT-Stabilisation geworfen. Genauer wird erst das vorgegebene Problem mittels der RBF-FD-Methode gelöst und dann durch das FCT-Verfahren stabilisiert. Der allgemeinene Simulationsalgorithmus ist in Abbildung \ref{fig:algorithmus} in einem Pfeildiagramm graphisch dargestellt. Hierbei sei $u_{initial}$  die Anfangsbedingung, $u_{pure}$ die Lösung des Verfahrens höherer Ordnung, $u_{diff}$ die Lösung nach der Addition der künstlichen Diffusion und $u_{stab}$ die stabilisierte Lösung am Ende des FCT-Schema. In der Zeitschleife wird somit pro Zeitschritt die Lösungen $u_{pure}$, $u_{diff}$ und $u_{stab}$ nacheinander errechnet.
\begin{figure}[H]
\centering
   \begin{tikzpicture}[>=stealth]
    \node (at) at (0,0) {$u_{initial}$};
    \node[below=2cm of at] (at-high) {$u_{pure}$};
    \draw[->] (at) -- (at-high) node[midway,right,font=\small] {start};
    \node[below right=2cm and 1cm of at-high] (at-diff) {$u_{diff}$};
    \node[below=1.25cm of at-high] (at-loop) {loop};
    \node[below left=2cm and 1cm of at-high] (at-stab) {$u_{stab}$};
    \draw[->,shorten >=3pt] (at-high) to[bend left] (at-diff);
    \draw[->,shorten >=3pt] (at-diff) to[bend left] (at-stab);
    \draw[->,shorten >=3pt] (at-stab) to[bend left] (at-high);
  \end{tikzpicture}
\caption{FCT-Algorithmus}
\label{fig:algorithmus}
\end{figure}
\noindent
Nach dem Algorithmus ist zu erwarten, dass die stabilisierte Lösung eine bessere Approximation der analytischen Lösung ist, als die Lösungen der RBF-FD-Methode ohne Stabilisation. Um diese Hypothese zu belegen werden im Folgenden die Methoden anhand der Transportgleichung, welche im Kapitel \ref{sec:transglei} vorgestellt wurde,  auf zwei unterschiedliche Probleme angewendet. Hierbei wird als Transport die Rotation einer gegebenen Funktion im Einheitsquadrat $\Omega=[0,1]^2$ um den Punkt $(0.5,0.5)$ simuliert. Als Auswertungspunkte werden in diesem Abschnitt Punkte mit äquidistanten Abständen genommen, welches als ein äquidistantes Gitter interpretiert werden kann. Hierbei sollte zur Kenntnis genommen werden, dass in diesem Unterabschnitt der Vorteil des RBF-FD-Verfahren, das gitterfreie Lösen, nicht im Fokus steht, sodass hier einfachhaltshalber diese Positionierung gewählt wurde.\\\\
\textbf{1. Versuch}\\\\
Im ersten Versuch werden die unterschiedlichen Lösungen anhand der in Abbildung \ref{fig:initial1} dargestellten Anfangsbedingung für unterschiedliche Parameter gegenübergestellt.
\begin{figure}[H]
\centering
 \includegraphics[width=0.8\textwidth]{figures/T_pi/initial_solution_n_100.png}
\caption{Anfangsbedingung}
\label{fig:initial1}
\end{figure}
\noindent
Es werden die Lösungen für $N=400$ Unbekannten und eine Rotation um $\alpha=\frac{\pi}{2}$ analysiert. Hierbei sind die Lösungen in den Abbildungen \ref{fig:pur_20} - \ref{fig:stab_20} graphisch dargestellt, wobei zu beachten ist, dass durch die graphische Ausgabe die Lösung für einige Beispiele nicht exakt wiedergespiegelt wird. Dies erkennt man beispielsweise an der Abbildung \ref{fig:pur_20}, in der es so aussieht, als ob manche Knotenwerte gleich $-1$ sind, obwohl dies nicht der Fall ist.\\
Die unterschiedlichen Eigenschaften der drei Lösungen sind in den Abbildungen klar wiederzufinden. Das RBF-FD-Verfahren weist eine stark oszillierende Lösung im ganzen Gebiet $\Omega$ auf, womit die Positivitätsbedingung nicht eingehalten werden kann. Dahingegen ist die Lösung $u_{diff}$ stark diffusiv und die Maximalwerte verringern sich in diesem Beispiel von $1$ auf $311.795635768968e-003$. Die stabilisierte Lösung sieht auf dem ersten Blick als beste Approximation der drei Lösungen aus, da diese nicht oszillierend und nicht zu starke diffusive Eigenschaften ausweist.
\begin{figure}[H]
\centering
 \includegraphics[width=0.8\textwidth]{figures/T_pi_2/numerical_solution_high_n_20.png}
\caption{Lösung $u_{pure}$ für $N=400$ und $\alpha=\frac{\pi}{2}$}
\label{fig:pur_20}
\end{figure}
\begin{figure}[H]
\centering
\includegraphics[width=0.8\textwidth]{figures/T_pi_2/numerical_solution_diff_n_20.png}
\caption{Lösung $u_{diff}$ für $N=400$ und $\alpha=\frac{\pi}{2}$}
\label{fig:diff_20}
\end{figure}
\begin{figure}[H]
\centering
\includegraphics[width=0.8\textwidth]{figures/T_pi_2/numerical_solution_stab_high_n_20.png}
\caption{Lösung $u_{stab}$ für $N=400$ und $\alpha=\frac{\pi}{2}$}
\label{fig:stab_20}
\end{figure}
\noindent
Um dies zu prüfen, wurden die Fehler der einzelnen Lösungen zu der analytischen Lösung $u_{ana}$ ermittelt und in Tabelle \ref{tab:fehler_20_pi/2} zusammengefasst. Zur Fehlerberechnung wurde die diskrete L2-Norm 
\begin{align}
 \|v\|_{l^2(\Omega)}:=\sqrt{\frac{1}{N}\sum\limits_{i=1}^Nv_i^2}\label{eq:discretel2norm}
\end{align}
verwendet, da diese das allgemeine Mittel der Fehler auf dem gesamten Gebiet $\Omega$ ermittelt und nicht von Bereichen, welche numerisch schwierig zu approximieren sind, abhängig ist.
\begin{table}[H]
\centering
\begin{tabular}[\textwidth]{|c|c|}
\hline
 Fehlerfunktion & Fehlerwerte\\
\hline
$\|u_{ana}-u_{pure}\|_{l^2(\Omega)}$ & $205.649439898788e-003$\\
%\hline
$\|u_{ana}-u_{diff}\|_{l^2(\Omega)}$ & $199.979489340381e-003$\\
%\hline
$\|u_{ana}-u_{stab}\|_{l^2(\Omega)}$ & $144.092770573993e-003$\\
\hline
\end{tabular}
\caption{Fehlerwerte für $N=400$ und $\alpha=\frac{\pi}{2}$}
\label{tab:fehler_20_pi/2}
\end{table}
\noindent
Man erkennt, dass der Fehler für das Verfahren mit FCT-Stabilisation mit einem Fehlerwert von $144.092770573993e-003$ niedriger ist als der Fehlerwert ohne Stabilisation mit $205.649439898788e-003$. Dies liegt an den Oszillationen der RBF-FD-Lösung, wodurch mehr Summanden in der Normberechnung ungleich Null sind und somit die gesamte Summe größer wird. Des Weiteren erkennt man, das der Fehler der diffusiven Lösung minimal besser ist als der Fehler des Verfahrens höherer Ordnung, obwohl die Maximalwerte in diesem Beispiel mit $N=400$ stark sinken.\\
Um das allgemeine Verhalten und die allgemeine Konvergenz der stabilisierten Lösung graphisch zu zeigen werden im Folgenden die Parameter $N$ und $\alpha$ jeweils variiert.\\
Zuerst wird die Anzahl der Unbekannten für eine feste Rotation $\alpha=\pi$ verändert. Hierzu wird die stabilisierte Lösung $u_{stab}$ in den Abbildungen \ref{fig:stab_5_10_pi} - \ref{fig:stab_80_100_pi} für eine Anzahl von Unbekannten $N$ von 25 bis 10000 gegenübergestellt. Man erkennt eine grahische Konvergenz der stabiliserten Lösung ohne Oszillationen. Um diese Hypothese mathematisch zu verifizieren, sind die Fehlerwerte der Fehlerfunktion $u_{ana}-u_{stab}$ in der diskrete L2-Norm~\eqref{eq:discretel2norm} ermittelt und in der Tabelle \ref{tab:fehler_N_pi}  eingetragen worden. Man erkennt ein Konvergenzverhalten des Fehlers gegen 0, welches mit der graphischen Auswertung übereinstimmt.
\begin{figure}[H]
\hspace{0.1cm}
\begin{minipage}{0.4\textwidth}
\includegraphics[width=1.2\textwidth]{figures/T_pi/numerical_solution_stab_high_n_5.png}
\end{minipage}
\hspace{1.4cm}
\begin{minipage}{0.4\textwidth}
\includegraphics[width=1.2\textwidth]{figures/T_pi/numerical_solution_stab_high_n_10.png}
\end{minipage}
\caption{Lösung $u_{stab}$ für $N=25$ und $N=100$, $\alpha=\pi$}
\label{fig:stab_5_10_pi}
\end{figure}

\begin{figure}[H]
\hspace{0.1cm}
\begin{minipage}{0.4\textwidth}
\includegraphics[width=1.2\textwidth]{figures/T_pi/numerical_solution_stab_high_n_20.png}
\end{minipage}
\hspace{1.4cm}
\begin{minipage}{0.4\textwidth}
\includegraphics[width=1.2\textwidth]{figures/T_pi/numerical_solution_stab_high_n_40.png}
\end{minipage}
\caption{Lösung $u_{stab}$ für $N=400$ und $N=3600$, $\alpha=\pi$}
\label{fig:stab_20_40_pi}
\end{figure}

\begin{figure}[H]
\hspace{0.1cm}
\begin{minipage}{0.4\textwidth}
\includegraphics[width=1.2\textwidth]{figures/T_pi/numerical_solution_stab_high_n_80_pi.png}
\end{minipage}
\hspace{1.4cm}
\begin{minipage}{0.4\textwidth}
\includegraphics[width=1.2\textwidth]{figures/T_pi/numerical_solution_stab_high_n_100_pi.png}
\end{minipage}
\caption{Lösung $u_{stab}$ für $N=3600$ und $N=10000$, $\alpha=\pi$}
\label{fig:stab_80_100_pi}
\end{figure}
\noindent
\begin{table}[H]
\centering
\begin{tabular}[\textwidth]{|c|c|c|}
\hline
  & $\|u_{ana}-u_{stab}\|_{l^2(\Omega)}
$ & $\|u_{ana}-u_{pure}\|_{l^2(\Omega)}$\\
\hline
$N=5^2$ & $226.370914354391e-003$ & $193.016969143063e-003$\\
%\hline
$N=10^2$ & $175.779056969637e-003$ & $218.622664823324e-003$\\
%\hline
$N=20^2$ & $154.026021594423e-003$ & $207.422627092587e-003$\\
\hline
$N=40^2$ & $118.716820522281e-003$ & $163.790991728737e-003$\\
%\hline
$N=80^2$ & $103.237909256097e-003$ & $119.284504788009e-003$\\
%\hline
$N=100^2$ & $95.2290159199564e-003$ & $109.636614814776e-003$\\
\hline
\end{tabular}
\caption{Fehlerwerte für beliebiges $N$ und festem $\alpha=\pi$}
\label{tab:fehler_N_pi}
\end{table}
\noindent
Betrachtet wird nun der Versuch, indem die Rotation variabel und die Nummer der Unbekannte vorgegeben ist. Die numerischen Ergebnisse wurden in den Abblidungen \ref{fig:stab_80_pi} und \ref{fig:stab_80_2pi} für Rotationen von $\alpha=\frac{\pi}{2}$ bis $\alpha=2\pi$ für $N=6400$ Unbekannten graphisch dargestellt. Man erkennt eine Verschlechterung der stabilisierten Lösung $u_{stab}$, je länger die Rotation und somit die Simulation andauert. Dies ist vor allem an dem Einschnitt im Zylinder zu erkennen, da nach einer einer Rotation um  $2\pi$ der rechte Flügel stark abnimmt. Diese Verschlechterung spiegelt sich auch in den evaluierten, diskreten Fehlern, welche in Tabelle \ref{tab:fehler_N_rotation} notiert sind, dar. Man erkennt eine minimale Anhebung des dikreten $L_2$-Fehlers für längere Simulationen.
\begin{figure}[H]
\hspace{0.1cm}
\begin{minipage}{0.4\textwidth}
\includegraphics[width=1.2\textwidth]{figures/n_80_pi_variable/T_pi_2/numerical_solution_stab_high_n_80_pi_2.png}
\end{minipage}
\hspace{1.4cm}
\begin{minipage}{0.4\textwidth}
\includegraphics[width=1.2\textwidth]{figures/n_80_pi_variable/T_pi/numerical_solution_stab_high.png}
\end{minipage}
\caption{Lösung $u_{stab}$ für $\alpha=\frac{\pi}{2}$ und $\alpha=\pi$, $N=6400$}
\label{fig:stab_80_pi}
\end{figure}
\begin{figure}[H]
\hspace{0.1cm}
\begin{minipage}{0.4\textwidth}
\includegraphics[width=1.2\textwidth]{figures/n_80_pi_variable/T_3_pi_2/numerical_solution_stab_high.png}
\end{minipage}
\hspace{1.4cm}
\begin{minipage}{0.4\textwidth}
\includegraphics[width=1.2\textwidth]{figures/n_80_pi_variable/T_2_pi/numerical_solution_stab_high.png}
\end{minipage}
\caption{Lösung $u_{stab}$ für $\alpha=\frac{3\pi}{2}$ und $\alpha=2\pi$, $N=6400$}
\label{fig:stab_80_2pi}
\end{figure}
\noindent
\begin{table}[H]
\centering
\begin{tabular}[\textwidth]{|c|c|c|}
\hline
  & $\|u_{ana}-u_{stab}\|_{l^2(\Omega)}$ & $\|u_{ana}-u_{pure}\|_{l^2(\Omega)}$\\
\hline
$\frac{\pi}{2}$ & $90.1222673276692e-003$ & $114.640811859429e-003$\\
%\hline
$\pi$ & $103.237909256097e-003$ & $119.284504788009e-003$\\
\hline
$\frac{3\pi}{2}$ & $111.479510855035e-003$ & $124.691327025648e-003$\\
%\hline
$2\pi$ & $117.832000511267e-003$ & $133.401777622865e-003$\\
\hline
\end{tabular}
\caption{Fehlerwerte für beliebiges $\alpha$ und festem $N=6400$}
\label{tab:fehler_N_rotation}
\end{table}
\noindent
Abschließend wird nun ein letzter Test durchgeführt, um die universelle Anwendbarkeit des Verfahrens hervorzuheben. Bei der RBF-FD-Methode mit FCT-Stabilisation gibt es allgemein keine Anfangswerteinschränkungen. Um dies graphisch zu verifizieren wird im Folgenden das Gesicht von Carl Friedrich Gauß geplottet und dann um den Punkt $(0.5,0.5)$ mittels dem zugrundeliegenden Verfahren rotiert.\\
Die Anfangsbedingung von vorne und von oben betrachtet sind in Abbildung \ref{fig:init_gauss} für $10000$ Unbekannten visualisiert. Man erkennt keine klare Struktur der Anfangsbedingung und ist somit willkürlich ausgewählt worden. Nun wird das Potrait von Gauß um $2\pi$ rotiert und die Zwischenergebnisse in Abbildung \ref{fig:gauss_pi} und \ref{fig:gauss_2pi} abgebildet. Nach der Simulation kann man nicht mehr das komplette Gesicht erkennen, doch signifikante Stellen, wie beispielsweise die Augen oder die Nase, sind noch erkennbar. Dies zeigt nun, dass durch diese stabilisierte Methodik jegliche Probleme, beziehungsweise jegliche Anfangsbedingungen, approximierbar sind.
\begin{figure}[H]
\begin{minipage}{0.4\textwidth}
\includegraphics[width=1.2\textwidth]{figures/Gesicht/n_100/Pi_2/initial_solution_n_100.png}
\end{minipage}
\hspace{1.4cm}
\begin{minipage}{0.4\textwidth}
\includegraphics[width=1.2\textwidth]{figures/Gesicht/n_100/Pi_2/initial_solution_n_100_top.png}
\end{minipage}
\caption{Anfangsbedingung $u_{initial}$, $N=10000$}
\label{fig:init_gauss}
\end{figure}
\begin{figure}[H]
\begin{minipage}{0.45\textwidth}
 \includegraphics[width=1.2\textwidth]{figures/Gesicht/n_100/Pi_2/numerical_solution_stab_high_gesicht_n_100_pi_2.png}
\vspace{-0.4cm}
\end{minipage}
\begin{minipage}{0.45\textwidth}
 \includegraphics[width=1.2\textwidth]{figures/Gesicht/n_100/1_Pi/numerical_solution_stab_high_gesicht_n_100_pi.png}
\vspace{-0.4cm}
\end{minipage}
\caption{Lösung: $u_{stab}$ für $\alpha=\frac{\pi}{2}$ und $\alpha=\pi$, $N=10000$}
\label{fig:gauss_pi}
\end{figure}
\vspace{-0.4cm}
\begin{figure}[H]
\begin{minipage}{0.45\textwidth}
 \includegraphics[width=1.2\textwidth]{figures/Gesicht/n_100/3_Pi_2/numerical_solution_stab_high_top2.png}
\vspace{-0.4cm}
\end{minipage}
\begin{minipage}{0.45\textwidth}
 \includegraphics[width=1.2\textwidth]{figures/Gesicht/n_100/2_Pi/numerical_solution_stab_high_gesicht_n_100_2_pi.png}
\vspace{-0.4cm}
\end{minipage}
\caption{Lösung: $u_{stab}$ für $\alpha=\frac{3\pi}{2}$ und $\alpha=2\pi$, $N=10000$}
\label{fig:gauss_2pi}
\end{figure}
\pagebreak
\section{Zusammenfassung und Ausblick}
In dieser Arbeit wurden allgemein zwei numerische Konzepte vorgestellt: die RBF-FD-Methode und die FCT-Stabilisierung. Beide Verfahren wurden jeweils theoretisch analysiert und dann anhand der Transportgleichung numerisch getestet.\\
Mit der RBF-FD-Methodik können partielle Differentialgleichung ohne ein fest vorgegebenes Gitter gelöst werden, sodass durch dieses Verfahren komplexere Gebiete besser approximiert werden können.\col{bla}\\
Als zweites wurde im Kapitel \ref{sec:FCT} ein genauerer Blick auf das FCT-Stabilisationsverfahren geworfen. Hierbei stellte sich heraus, dass durch Anwendung dieses Schemas die Approximation der Lösung durch ein numerisches Verfahren stark stabilisiert wurde und alle künstlichen Oszillationen stark gedämpft wurden. Des Weiteren kann durch dieses Verfahren die Positivität der Lösung sichergestellt werden, welches für viele physikalische Beispiele von hoher Relevanz ist. In Kapitel \ref{sec:num_res_FCT} wurde dieses Verfahren anhand der numerische Approximation durch das RBF-FD-Verfahren vorgestellt und unterschiedliche Resultate vorgestellt. An dieser Stelle wurde mit unterschiedlichen Bedingungen das FCT-Verfahren getestet und man erkennt, dass dieses Verfahren schon für eine niedrige Anzahl von Unbekannten von 10000 schon eine gute Näherung der analytischen Lösung liefert. Dennoch ist der verwendete Zahlenbereich von 25-10000 Unbekannten für heutige Problemstellungen viel zu niedrig. Deswegen  sollte man das Verfahren für eine effizientere Programmiersprache, wie C++, implementieren, um eine höhere Anzahl von Unebkannten realisieren zu können und somit eine höhere Genauigkeit zur analytischen Lösung zubekommen.
\newpage
\pagenumbering{Roman}
\setcounter{page}{3}
\appendix
\bibliography{Bachlorarbeit}
\bibliographystyle{unsrt}
\end{document}
